\chapter{Quantum Latin Squares}

In this chapter we will explain quantum Latin squares with examples at later stage, First we discuss Latin squares, their historical context, and their relationship with Quasi-group and Cayley table(\textit{comment: Yet I need to study more about quasi-group and cayley table}). 

In this thesis the notations are as follows, the set of natural numbers is denoted $\mathbb{N}$ and the set of natural numbers including zero is denoted $\mathbb{N}_0$.
The set of complex numbers is denoted $\mathbb{C}$.
The set $\{ k \in \mathbb{N}_0 \mid k < n \}$ of natural numbers (including zero) less than $n$ is denoted $[n]$.
We index the rows of an array in order with the elements of $[n]$ in their natural order, likewise for the columns.


\subsubsection{Latin Squares}

Latin squares were discussed by Euler in 1782~\cite{macneish1922EulerSquares} and Fisher in 1935~\cite{fisher1960DesignExperiments}.
The latter used Latin squares to design modern hypothesis testing experiments.
Since then Latin squares have been used in number theory, statistical analysis, game theory, computer science, coding theory, cryptography, and quantum information theory~\cite{keedwell2015LatinSquaresTheir}.

\begin{definition}[Latin Square]
  Let $S$ be a set of order $n \in \mathbb{N}$.
  An $n \times n$ array is called a Latin square of order $n$ if each element of $S$ appears exactly once in each row and exactly once in each column of the array.
\end{definition}

Suppose we construct a Latin square of order $n$ with elements from $[n]$.
Then each row and each column contains the elements of $[n]$ in some order. Here $n = 4$ and $[n]=\{0,1,2,3\}$. 

\begin{figure}[H]
  \centering
  \begin{tabular}{|c|c|c|c|}
    \hline
    0 & 1 & 2 & 3 \\
    \hline
    1 & 0 & 3 & 2 \\
    \hline
    2 & 3 & 0 & 1 \\
    \hline
    3 & 2 & 1 & 0 \\
    \hline
  \end{tabular}
  \caption{Latin Square}
  \label{fig:latin_square}
\end{figure}

 The Figure~\ref{fig:latin_square} is a Latin square of order 4. Each row and each column contain unique elements, there is not any repetition of elements in same row and same column.

\subsubsection{Quantum Latin squares}

Quantum Latin squares are combinatorial structures analogous to Latin squares.
A quantum Latin square is an array of $n$ rows and $n$ columns containing elements of $C^n$ such that each row and each column forms an orthonormal basis for $C^n$.
As above, we define the inner product by $\braket{a | b} = \sum_{i=0}^{n} a_i^* b_i$ for $a$ and $b$ in $\mathbb{C}^n$~\cite{zauner2011QuantumDesignsFoundations}.


$$
\begin{array}{|c|c|c|c|}
  \hline
  \ket{0} & \ket{1} & \ket{2} & \ket{3} \\
  \hline
  \ket{1} & \ket{0} & \ket{3} & \ket{2} \\
  \hline
  \ket{2} & \ket{3} & \ket{0} & \ket{1} \\
  \hline
  \ket{3} & \ket{2} & \ket{1} & \ket{0} \\
  \hline
\end{array}
$$


$$
\begin{array}{|c|c|c|c|}
  \hline
  \ket{0} & \ket{1} & \ket{2} & \ket{3} \\
  \hline
  \frac{1}{\sqrt{2}}(\ket{1}-\ket{2}) & \frac{1}{\sqrt{5}}(i\ket{0}+2\ket{3}) & \frac{1}{\sqrt{5}}(2\ket{0}+i\ket{3}) & \frac{1}{\sqrt{2}}(\ket{1}+\ket{2}) \\
  \hline
  \frac{1}{\sqrt{2}}(\ket{1}+\ket{2}) & \frac{1}{\sqrt{5}}(2\ket{0}+i\ket{3}) & \frac{1}{\sqrt{5}}(i\ket{0}+2\ket{3}) & \frac{1}{\sqrt{2}}(\ket{1}-\ket{2}) \\
  \hline
  \ket{3} & \ket{2} & \ket{1} & \ket{0} \\
  \hline
\end{array}
$$ 


$$
\braket{\psi_1 \mid \psi_2} = \sum_{i=0}^{n-1} \psi_1^* \psi_2
$$

\subsection{Normalized}

\begin{align}
\ket{\psi} &= \frac{1}{\sqrt{2}}(\ket{1}-\ket{2}) \\
\ket{\psi} &\rightarrow \begin{bmatrix} 0 \\ \frac{1}{\sqrt{2}} \\ -\frac{1}{\sqrt{2}} \\ 0 \end{bmatrix} \\
\Rightarrow \bra{\psi} &= \begin{bmatrix} 0 & \frac{1}{\sqrt{2}} & -\frac{1}{\sqrt{2}} & 0 \end{bmatrix} \\
\Rightarrow \braket{\psi \mid \psi} &= \begin{bmatrix} 0 & \frac{1}{\sqrt{2}} & -\frac{1}{\sqrt{2}} & 0 \end{bmatrix} \begin{bmatrix} 0 \\ \frac{1}{\sqrt{2}} \\ -\frac{1}{\sqrt{2}} \\ 0 \end{bmatrix} \\
&= 1
\end{align}



\subsection{Orthogonal}

\begin{align}
\ket{\psi_1} &\rightarrow \begin{bmatrix} 0 \\ \frac{1}{\sqrt{2}} \\ -\frac{1}{\sqrt{2}} \\ 0 \end{bmatrix} \\
\ket{\psi_2} &\rightarrow \begin{bmatrix} \frac{i}{\sqrt{5}} \\ 0 \\ 0 \\ \frac{2}{\sqrt{5}} \end{bmatrix} \\
\Rightarrow \bra{\psi_1} &= \begin{bmatrix} 0 & \frac{1}{\sqrt{2}} & -\frac{1}{\sqrt{2}} & 0 \end{bmatrix} \\
\Rightarrow \braket{\psi_1 \mid \psi_2} &= \begin{bmatrix} 0 & \frac{1}{\sqrt{2}} & -\frac{1}{\sqrt{2}} & 0 \end{bmatrix} \begin{bmatrix} \frac{i}{\sqrt{5}} \\ 0 \\ 0 \\ \frac{2}{\sqrt{5}} \end{bmatrix} \\
&= (0)(\frac{1}{\sqrt{5}}) + \left(\frac{1}{\sqrt{2}}\right)(0) + (-\frac{1}{\sqrt{2}})(0) + (0)(\frac{2}{\sqrt{5}}) \\
&= 0
\end{align}


\section{Code}


You might consider including a section describing your algorithm and code.
The minted package is great for displaying short sections of code, as in 
Listing \ref{code:tuesday}.



\begin{listing}[H]
  \begin{minted}{python}
    # Ian McLoughlin, 2018-02-01
    # Is it Tuesday?

    import datetime

    if datetime.datetime.today().weekday() == 1:
      print("Yay! It is Tuesday.")
    else:
      print("Unfortunately it is not Tuesday.")
  \end{minted}
  \caption{Is it Tuesday?}
  \label{code:tuesday}
\end{listing}

\lipsum[10-15]