%!TEX root = ../thesis.tex

\chapter{Introduction}

In the introduction, you should describe what your thesis is about, how the
thesis is organised, and what the reader can expect as they read down through
it.
The most important aspect of the introduction is to set the context for the
rest of the thesis.
You should sign-post for the reader the most important parts of your work and
where they appear in the document.
Don't forget to use any necessary references~\cite{einstein}.

\lipsum[1-2]

\section{Sections}
You might use sections to break the introduction down into easier to read
chunks.
\lipsum[1-2]

\subsection{Subsections}
You might even use subsections if you really need to.
There are also subsubsections in \LaTeX, but be careful not to overdo it.

\lipsum[1-2]