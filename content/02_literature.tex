\chapter{Literature}
\label{section:literature}

\subsection{Quantum bits}

The fundamental concept of the classical computation and classical information is \textit{bit}. A bit is the smallest unit of information in classical computing and can take one of two values: 0 or 1. It is the basic building block of classical computers and is used to represent and store information in binary form. However, in quantum computing, the basic unit of information is called a \textit{quantum bit} or \textit{qubit}. %A qubit can exist in a superposition of states, meaning it can be in a state of 0, 1, or both simultaneously. This property allows quantum computers to perform certain calculations much faster than classical computers.
Schumacher coined the term quantum bit or qubit in 1995~\cite{schumacher1995QuantumCoding} as a two-level quantum system in which information is encoded using two orthogonal states. Just like its classical cousin, the qubit can take a value of either 0 or 1. However, unlike a classical bit, a qubit can exist in a superposition of states, meaning it can be in a state of 0, 1, or both simultaneously. 
Physically, qubits can be represented as any two-level quantum systems such as
\begin{itemize}
  \item The spin of a particle in a magnetic field where up means 0 and down means 1 or
  \item The polarization of a single photon where horizontal polarization means 1 and vertical polarization means 0.
\end{itemize}

We can make a quantum computer out of light as well. In both cases 0 and 1 are the only possible states. Geometrically, qubits can be visualized using a shape called the Bloch sphere, an instrument named after Swiss physicist Felix Bloch as in figure~\ref{fig:blochSphere}.
% \begin{tikzpicture}[line join = round, line cap = round]
%   \pgfmathsetmacro{\r}{2} % radius of the sphere
%   \draw (0,0) circle (\r cm); % draw the sphere
%   \draw[-latex] (0,0) -- (\r+0.5,0); % draw x-axis
%   \draw[-latex] (0,0) -- (0,\r+0.5); % draw y-axis
%   \draw[-latex] (0,0) -- (-\r-0.5,0); % draw -x-axis
%   \draw[-latex] (0,0) -- (0,-\r-0.5); % draw -y-axis
%   \node[above] at (0,\r+0.5) {$|0\rangle$}; % label for |0>
%   \node[below] at (0,-\r-0.5) {$|1\rangle$}; % label for |1>
%   \node[right] at (\r+0.5,0) {$|+\rangle$}; % label for |+>
%   \node[left] at (-\r-0.5,0) {$|-\rangle$}; % label for |->
%   \draw[fill=black] (45:\r) circle (2pt); % draw a dot at angle \theta
%   \node[right] at (45:\r) {$|\psi\rangle$}; % label for |\psi>
% \end{tikzpicture}


\begin{figure}[H]
  \centering
\tdplotsetmaincoords{60}{120}
\begin{tikzpicture}[tdplot_main_coords]
  \pgfmathsetmacro{\r}{2} % radius of the sphere
  \shade[ball color = gray!40, opacity = 0.5] (0,0) circle (\r cm); % draw the sphere
  \draw[-latex] (0,0,0) -- (\r+1.25,0,0); % draw x-axis
  \draw[-latex] (0,0,0) -- (-\r-1.25,0,0); % draw x-axis
  \draw[-latex] (0,0,0) -- (0,\r+0.5,0); % draw y-axis
  \draw[-latex] (0,0,0) -- (0,-\r-0.5,0); % draw y-axis
  \draw[-latex] (0,0,0) -- (0,0,\r+0.5); % draw z-axis
  \draw[-latex] (0,0,0) -- (0,0,-\r-0.5); % draw z-axis
  \node[above] at (0,0,\r+0.5) {$|0\rangle$}; % label for |0>
  \node[below] at (0,0,-\r-0.5) {$|1\rangle$}; % label for |1>
  \node[right] at (\r+1.75,0,0) {$|+\rangle$}; % label for |+>
  \node[left] at (-\r-1.75,0,0) {$|-\rangle$}; % label for |->
  \draw[fill=black] (45:\r) circle (2pt); % draw a dot at angle \theta
  \node[right] at (45:\r) {$|\psi\rangle$}; % label for |\psi>
\end{tikzpicture}
\caption{Bloch Sphere representation for a Qubit.}
\label{fig:blochSphere}
\end{figure}



Mathematically, the qubit is treated as a mathematical object and the state of a qubit is treated as a vector in a two-dimensional complex vector space, as in equation~\ref{eq:qubit_state}
\begin{equation}
\ket{\psi} = \alpha \ket{0} + \beta \ket{1}
\label{eq:qubit_state}
\end{equation}

where $\alpha$ is the probability amplitude of the state $\ket{0}$ and $\beta$ is the probability amplitude of the state $\ket{1}$.

A probability amplitude is a complex number $c$ whose squared magnitude $\vert c \vert^2$ gives the probability of finding the quantum system in its associated orthogonal state.
The probabilities $|\alpha|^2$ and $|\beta|^2$ must therefore sum to one, so qubits are in that sense normalized.
We define the inner product $\braket{a | b} = \sum_{i=0}^{n} a_i^* b_i$ for $a$ and $b$ in $\mathbb{C}^n$, leading to a Hilbert space.
The states $\ket{0}$ and $\ket{1}$ form a basis for the Hilbert space.
Thus, we can write
$$\ket{0} = \begin{bmatrix} 1 \\ 0 \end{bmatrix} \, \textrm{and} \, \ket{1} = \begin{bmatrix} 0 \\ 1 \end{bmatrix}$$



\subsubsection{Systems of Qubits}
Two or more qubits can be combined in a single system.
Where $\mathcal{H}_1$ and $\mathcal{H}_2$ are two Hilbert spaces, the tensor product $\mathcal{H}_1 \otimes \mathcal{H}_2$ is the Hilbert space of the combined system.
In a combined system, qubits can become entangled in the sense that measurements of the qubits become correlated.
Suppose we denote the four orthogonal computational states of a two qubit system as $\{\ket{00}, \ket{01}, \ket{10}, \ket{11}\}$.
The system is correlated when the state vector $\alpha_{00} \ket{00} + \alpha_{01} \ket{01} + \alpha_{10} \ket{10} + \alpha_{11} \ket{11}$ cannot be decomposed into a tensor product of two independent state vectors, one for each qubit.

\subsubsection{Physical and Logical Qubits}
We distinguish between physical qubits and logical qubits.
A physical qubit is a physical quantum object with two levels.
Various physical systems can be used to realize physical qubits such as the two different polarizations of a photo.
A logical qubit is a mathematical object used to model a physical qubit.
Where physical qubits are fragile, a logical qubit might be encoded by several physical qubits.

\subsection{Unitary Operators and Unitary Error Bases (UEB)}
Unitary operators are also discussed in 1977 in book quantum Mechanics by Cohen~\cite{cohen1977quantummechanics1st}.

\subsubsection{Operators}

In quantum mechanics, operators are used to represent physical observables(i.e. energy, momentum, position, etc.) and mathematically they these physical observables are represented by Hermitian operators. For instance, the operator corresponding to energy in the Hamiltonian operator~\ref{eq:hamiltonian}.
\begin{equation}
\hat{H} = \sum_{i=1}^{N}(-\frac{\hbar^{2}}{2m_{i}}\nabla_{i}^{2}+V_{i})
\label{eq:hamiltonian}
\end{equation}
where i is an index over all particles of the system. The result $\hat{H}$ is the Hamiltonian equal to the individual sum of all particles of the system. The average value of an observable A, operator denoted by $\hat{A}$ for a quantum molecular state $\psi{(r)}$ is given by the 'expectation value' formula:

\begin{equation}
\langle A \rangle = \int \psi^{*}(r) A \psi(r) dr
\label{eq:expectation_value}
\end{equation}

In Equation \ref{eq:expectation_value}, we calculate the expectation value of the observable $A$ or operator $\hat{A}$. These operators show certain mathematical properties as summarized below:

\begin{itemize}
    \item The sum and difference of two operators $hat{A}$ and $\hat{B}$ are also operators, given by equation~\ref{eq:sum_diff_operator}.
        \begin{equation}
            (\hat{A} + \hat{B})f = \hat{A}f + \hat{B}f
            \label{eq:sum_diff_operator}
        \end{equation}
    \item The product of two operators is defined by 
        \begin{equation}
            (\hat{A}\hat{B})f = \hat{A}(\hat{B}f)
            \label{eq:product_operator}
        \end{equation}

    \item Two operators are equal if 
        \begin{equation}
            \hat{A}f = \hat{B}f
            \label{eq:equal_operator}
        \end{equation}
        for all functions $f$ in equation~\ref{eq:equal_operator}.

    \item The identity operator $\hat{I}$ is defined by equation~\ref{eq:identity_operator}.
        \begin{equation}
            \hat{I}f = f
            \label{eq:identity_operator}
        \end{equation}
          Identity element operator does not change anything when it operates on a function.
    \item The associative law also holds for operators as in equation~\ref{eq:associative_operator}.
        \begin{equation}
            \hat{A}(\hat{B}\hat{C}) = (\hat{A}\hat{B})\hat{C}
            \label{eq:associative_operator}
        \end{equation}
    \item The commutative law does not hold for operators. In general, $\hat{A}\hat{B} \neq \hat{B}\hat{A}$.  It is convenient to define the quantity as in equation~\ref{eq:commutative_operator}.
        \begin{equation}
            [\hat{A}, \hat{B}] = \hat{A}\hat{B} - \hat{B}\hat{A}
            \label{eq:commutator_operator}
        \end{equation}
          which is called commutator of $\hat{A}$ and $\hat{B}$. Note that the order matters, so that $[\hat{A}, \hat{B}] \eq -[\hat{B}, \hat{A}]$. If $\hat{A}$ and $\hat{B}$ commute, then $[\hat{A}, \hat{B}] = 0$.

  \end{itemize}

\begin{definition}[Unitary Error Basis]
A set $U$ of $n^2$ unitary $n \times n$ matrices is called a unitary error basis if and only if $U$ is orthonormal with respect to the inner product $\langle A, B \rangle = {\textrm{Tr}(A^\dagger B)} / n$ where $\textrm{Tr}(M)$ is the trace of the matrix $M$ and $M^\dagger$ is the conjugate transpose of $M$.
\end{definition}

\noindent Unitary error bases perform a primitive job in quantum error correcting codes and further applications of quantum information processing, such as teleportation and super-dense coding schemes~\cite{klappenecker2003UnitaryErrorBases}.

\subsection{Mutually Unbiased Bases (MUB)}

\begin{definition}[Mutually Unbiased Bases]
Two orthonormal bases $U$ and $V$ of $\mathbb{C}^n$ are mutually unbiased if and only if $|\braket{u|v}|^2 = \frac{1}{n}$ for all $u$ in $U$ and all $v$ in $V$.
\end{definition}

\noindent Mutually unbiased bases embed the complementary concept of quantum formalism that relates to~\cite{paterek2010ConnectionMutuallyUnbiased}.
Two orthonormal bases of Hilbert space are said to be mutually unbiased if the measurement of one state does not disclose the result of the other state~\cite{song2020ConstructionMutuallyUnbiased}.
Transition probability from any state of the first basis to any state of the second basis is independent of the two chosen states~\cite{song2020ConstructionMutuallyUnbiased}.


\subsubsection*{Mutually Unbiased Bases and Unitary Error Bases}
The idea of mutually unbiased bases was posed by Schwinger in 1960~\cite{schwinger1960UnitaryOperatorBases} and play an important role in quantum information theory~\cite{klappenecker2004ConstructionsMutuallyUnbiased}.
They are used in quantum state estimation~\cite{adamson2008ExperimentalQuantumState, grassl2005TomographyQuantumStates}, quantum cryptography protocols, and quantum error correcting codes~\cite{cerf2002SecurityQuantumKey}.

Mutually unbiased bases can be constructed using mutually orthogonal Latin squares and vice-versa~\cite{wocjan2005NewConstructionMutually, rao2010MutuallyOrthogonalLatin}.
The eigen-bases of the well-known Pauli operators $\hat{\sigma}_x$, $\hat{\sigma}_y$, and $\hat{\sigma}_z$ constitute mutually unbiased bases, where each vector in a given basis shares an identical degree of overlap with every vector in the other bases~\cite{paterek2009MutuallyUnbiasedBases}.
Note that the Pauli matrices are $d^2$ order unitary error bases with $d=2$ but they can be generalized to higher order dimensions~\cite{klappenecker2003UnitaryErrorBases}.
{\small
\begin{equation*}
    \sigma_I = \begin{bmatrix} 1 & 0  \\ 0 & 1  \end{bmatrix}, \;
    \sigma_x = \begin{bmatrix} 0 & 1  \\ 1 & 0  \end{bmatrix}, \;
    \sigma_y = \begin{bmatrix} 0 & -i \\ i & 0  \end{bmatrix}, \;
    \sigma_z = \begin{bmatrix} 1 & 0  \\ 0 & -1 \end{bmatrix}
\end{equation*}
}
Unitary error bases, also known as unitary operator bases, are the basic primitive in quantum information.
They were introduced in the mid 1990s, having great use in the area of constructing quantum error correcting codes~\cite{knill1996NonBinaryUnitary, knill1996GroupRepresentationsError, knill1996ConcatenatedQuantumCodes}. 
Quantum teleportation protocols, dense coding protocols and error correction can be precisely mathematically characterized as unitary error bases~\cite{klappenecker2003UnitaryErrorBases}.

Mutually unbiased bases and unitary error bases are fundamental primitives in quantum information theory.
They can be constructed using Latin squares and quantum Latin squares.
Considering their importance, Musto has introduced a way of constructing mutually unbiased bases using orthogonal quantum Latin squares, proving the reciprocal construction of quantum Latin squares from mutually unbiased bases~\cite{musto2017ConstructingMutuallyUnbiased}.

Unitary error bases can also be constructed from quantum Latin squares and a family of Hadamard matrices~\cite{vicary2016QuantumLatinSquares} via one of three known methods, as follows.
First, the \emph{shift and multiply} method uses classical Latin squares alongside a set of Hadamard matrices~\cite{werner2001AllTeleportationDense}.
Second, the \emph{Hadamard} method uses two mutually unbiased bases.
Finally, the \emph{Algebraic} method uses a finite group possessing a projective representation satisfying certain properties~\cite{knill1996GroupRepresentationsError}.


\section{Guidelines for writing}
A comprehensive literature review is a critical component of any academic research, including a Ph.D. thesis.
It serves several purposes, as follows.

\subsection{Establishing Context}
Provide background information on the topic of study to contextualize your research within existing knowledge and scholarship.
Identify key concepts, theories, and methodologies relevant to your research.

\subsection{Identifying Gaps and Opportunities}
Identify gaps, inconsistencies, or contradictions in the existing literature that your research aims to address.
Highlight areas where further research is needed or opportunities for innovation exist.

\subsection{Synthesizing Previous Research}
Summarize and synthesize findings from previous studies, organizing them thematically or chronologically.
Evaluate the strengths and weaknesses of previous research methodologies, data sources, and analytical approaches.
    
\subsection{Demonstrating Scholarly Engagement}
Demonstrate your familiarity with the current state of research in your field and your ability to critically evaluate and synthesize existing literature.
Establish your credibility as a researcher by showing that you are building upon established knowledge and contributing to ongoing scholarly conversations.
    
\subsection{Supporting Methodological Choices}
Justify your research methodology by explaining how it builds upon or diverges from previous approaches.
Discuss the suitability of different research methods and theoretical frameworks for addressing your research questions.
    
\subsection{Highlighting Significance and Contribution}
Emphasize the significance of your research by showing how it fills a gap, extends existing knowledge, or offers new insights.
Articulate the specific contribution your research makes to the field and how it advances scholarship.
    
\subsection{Citing Sources Appropriately}
Ensure that you accurately cite all sources consulted in your literature review according to the citation style guidelines required by your institution or discipline.

\subsection{Maintaining Focus and Organization}
Keep your literature review focused and well-organized, with clear sections or subsections devoted to different themes, theories, or methodologies.
Provide transitions and connections between different sections to maintain coherence and flow.

\subsection{Remaining Critical and Objective}
Remain critical and objective in your evaluation of previous research, acknowledging both strengths and limitations.
Avoid bias and strive to present a balanced and comprehensive overview of the literature.

\subsection{Updating and Revising}
Continuously update and revise your literature review as your research progresses and new relevant studies are published.
Be open to incorporating feedback from peers, advisors, and reviewers to strengthen your literature review.

By including these elements in your literature review, you can effectively situate your research within the broader scholarly conversation, demonstrate your expertise in the field, and lay the groundwork for your own empirical investigation.


\section{Maxwell's Equations}
Maxwell's equations are a set of four fundamental equations that describe the behavior of electric and magnetic fields, as well as their interaction with matter.
They are the cornerstone of classical electromagnetism and are essential for understanding a wide range of phenomena in physics and engineering.

\subsection{Gauss's Law for Electricity}
Gauss's Law for Electricity is one of the four fundamental equations in classical electromagnetism, formulated by Carl Friedrich Gauss.
It describes the relationship between the electric flux through a closed surface and the electric charge enclosed within that surface.
Mathematically, Gauss's Law for Electricity is expressed as:
\[
\oint_S \mathbf{E} \cdot d\mathbf{A} = \frac{Q_{\text{enc}}}{\varepsilon_0}
\]
Here's an explanation of the key components of Gauss's Law for Electricity:
\begin{itemize}
  \item $\oint_S$ represents a closed surface integral over a closed surface $S$. This means that we are summing the electric field ($\mathbf{E}$) over all infinitesimal areas ($d\mathbf{A}$) of the closed surface $S$.
  
  \item $\mathbf{E}$ is the electric field vector at each point on the surface. It represents the force experienced by a unit positive charge placed at that point.
  
  \item $d\mathbf{A}$ is a vector representing an infinitesimal area element of the surface. It is oriented perpendicular to the surface at each point.
  
  \item $Q_{\text{enc}}$ is the total electric charge enclosed within the closed surface $S$. This includes the sum of all positive and negative charges within the enclosed region.
  
  \item $\varepsilon_0$ is the permittivity of free space, a fundamental constant in electromagnetism. It represents the ability of a material to permit the formation of an electric field in response to an applied electric field.
\end{itemize}

In simpler terms, Gauss's Law for Electricity states that the total electric flux through a closed surface is proportional to the total electric charge enclosed within that surface, with the constant of proportionality being the permittivity of free space.
In other words, it quantifies how much electric field passes through a closed surface due to the presence of electric charges inside that surface.


\subsection{Gauss's Law for Magnetism}
Gauss's Law for Magnetism states that the magnetic flux through any closed surface is always zero.
Mathematically, it is expressed as:
\[
\oint_S \mathbf{B} \cdot d\mathbf{A} = 0
\]
Here's an explanation of the key components of Gauss's Law for Magnetism:
\begin{itemize}
  \item $\oint_S$ represents a closed surface integral over a closed surface $S$. This means that we are summing the magnetic field ($\mathbf{B}$) over all infinitesimal areas ($d\mathbf{A}$) of the closed surface $S$.
  
  \item $\mathbf{B}$ is the magnetic field vector at each point on the surface. Unlike electric fields, magnetic fields do not have sources or sinks (monopoles), so the magnetic flux through any closed surface is always zero.
  
  \item $d\mathbf{A}$ is a vector representing an infinitesimal area element of the surface. It is oriented perpendicular to the surface at each point.
\end{itemize}

In summary, Gauss's Law for Magnetism implies that there are no magnetic monopoles (isolated north or south poles), and the magnetic flux through any closed surface is always zero, indicating that magnetic field lines neither start nor end but always form closed loops.


\subsection{Faraday's Law of Induction}
Faraday's Law of Induction describes how a changing magnetic field induces an electromotive force (EMF) and hence an electric current in a conducting loop.
Mathematically, it is expressed as:
\[
\oint_C \mathbf{E} \cdot d\boldsymbol{\ell} = -\frac{d\Phi_B}{dt}
\]
Here's an explanation of the key components of Faraday's Law of Induction:
\begin{itemize}
  \item $\oint_C$ represents a closed path integral around a closed loop $C$. This means that we are summing the electric field ($\mathbf{E}$) around the closed loop $C$.
  
  \item $\mathbf{E}$ is the induced electric field within the conducting loop. It is created by a changing magnetic flux through the loop according to Faraday's law.
  
  \item $d\boldsymbol{\ell}$ is a vector representing an infinitesimal displacement along the closed loop $C$.
  
  \item $d\Phi_B/dt$ represents the rate of change of magnetic flux ($\Phi_B$) through the surface enclosed by the loop with respect to time. The negative sign indicates that the induced EMF and hence the induced electric field opposes the change in magnetic flux.
\end{itemize}

In summary, Faraday's Law of Induction states that a changing magnetic field induces an electric field and hence an electromotive force (EMF) in any closed conducting loop, producing an electric current in the loop. This phenomenon forms the basis of many practical devices, such as electric generators and transformers.


\subsection{Ampere's Circuital Law (with Maxwell's addition)}

Amp\`ere's Circuital Law relates the magnetic field around a closed loop to the electric current passing through the loop. With Maxwell's addition, it accounts for the displacement current, which arises from changing electric fields. Mathematically, it is expressed as:
\[
\oint_C \mathbf{B} \cdot d\boldsymbol{\ell} = \mu_0 \left( I_{\text{enc}} + \varepsilon_0 \frac{d\Phi_E}{dt} \right)
\]
Here's an explanation of the key components of Amp\`ere's Circuital Law (with Maxwell's addition):
\begin{itemize}
  \item $\oint_C$ represents a closed path integral around a closed loop $C$. This means that we are summing the magnetic field ($\mathbf{B}$) around the closed loop $C$.
  \item $\mathbf{B}$ is the magnetic field vector at each point along the closed loop $C$.
  \item $d\boldsymbol{\ell}$ is a vector representing an infinitesimal displacement along the closed loop $C$.
  \item $\mu_0$ is the permeability of free space, a fundamental constant in electromagnetism.
  \item $I_{\text{enc}}$ is the total current passing through the loop $C$. This includes both conduction current and displacement current.
  \item $\varepsilon_0$ is the permittivity of free space, another fundamental constant in electromagnetism.
  \item $\frac{d\Phi_E}{dt}$ represents the rate of change of electric flux ($\Phi_E$) through the surface enclosed by the loop with respect to time. This gives rise to the displacement current, which is included in Amp\`ere's law with Maxwell's addition.
\end{itemize}

In summary, Amp\`ere's Circuital Law with Maxwell's addition states that the magnetic field around a closed loop is proportional to the total current passing through the loop, including both conduction current and displacement current arising from changing electric fields. This law plays a crucial role in understanding the behavior of electromagnetic fields in various physical systems.



